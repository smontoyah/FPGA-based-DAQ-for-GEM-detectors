\documentclass[]{book}
\usepackage[english]{babel}
\usepackage{graphicx}

\begin{document}


\chapter*{System}

\subsection*{Front-end electronics}
\noindent Según la configuración de los electrodos de recolección de carga del detector, es posible sensar eventos en varios canales de lectura paralelamente, logrando una importante resolución espacial. Sin embargo, en este trabajo se desarrolla un sistema monocanal para la adquisición de datos de un detector GEM como versión mínima funcional. Los electrodos de salida del detector están conectados en paralelo y conducidos a una única conexión con un cable tipo X. Posteriormente, se conecta una resistencia en serie para permitir la medición de la caída de voltaje en sus terminales, lo que convierte la señal de corriente original del detector en una señal de voltaje que puede ser acondicionada.\\

%inlcuir fotografía de los terminales soldados y la resistencia en serie con el conector del cable

En particular, el sistema de preamplificación utilizado en este trabajo es el Ortec 142B, un dispositivo charge-sensitive diseñado para capacitancias de entrada de entre 100 y 400 pF. %inlcuir fotografía del preamp\\
%qué tantas especificaciones técnicas del preamp debería incluir aquí?

La señal de salida del preamplificador está caracterizada por... y teniendo en cuenta que el ADC tiene un rango de trabajo de [citar satasheet del ADC500]..., es necesario agregar una etapa de amplificación para imprimir en la señal una ganancia de x.\\

\end{document}