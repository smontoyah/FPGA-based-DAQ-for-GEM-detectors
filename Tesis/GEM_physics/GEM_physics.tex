\documentclass[]{book}
\usepackage[english]{babel}
\usepackage{graphicx}
\usepackage{amsmath}


\begin{document}
\chapter*{Overview on GEM Detectors}

En numerosos experimentos de física de partículas, los detectores gaseosos, incluidos los GEMs, desempeñan un papel crucial en la medición de partículas cargadas mediante la ionización de gases. Estos detectores permiten trazar con precisión las trayectorias de las partículas, especialmente en entornos con campos magnéticos. En comparación con los detectores de semiconductores, los detectores gaseosos suelen ser más económicos, especialmente en aplicaciones que requieren cubrir grandes volúmenes, y presentan menos material que pueda interferir con las partículas que atraviesan el detector [Sauli]. Dado que los detectores GEM son el enfoque principal de este trabajo, en el presente capítulo se explorarán los conceptos fundamentales de la física que rigen sus interacciones y funcionamiento.

\section{Electromagnetic interactions of charged particles with matter}

\section{Gaseous detectors}

\noindent La base para la detección de partículas en los detectores llenos de gas es la creación de cargas mediante ionización. Para la eficiencia en la formación de señales, también es crucial que las cargas no se pierdan por recombinación o adherencia mientras se desplazan hacia los electrodos.

\section{GEM detectors}




\end{document}