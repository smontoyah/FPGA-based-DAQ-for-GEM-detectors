\documentclass[]{book}
\usepackage[english]{babel}
\usepackage{graphicx}
\usepackage{amsmath}


\begin{document}
\chapter*{Teoría}
\section{GEM Physics}

% Tal es el caso de los detectores GEM. Tras las generaci´on de pares de electrones
% y iones en el interior de la c´amara de gas por su interacci´on con part´ıculas
% cargadas incidentes, Los electrones libres generados en la ionizaci´on primaria
% son atra´ıdos hacia los agujeros de la hoja de GEM debido al campo el´ectrico.
% Al pasar a trav´es de los agujeros, los electrones experimentan un fuerte campo
% el´ectrico local dentro de estos, lo que causa una avalancha de electrones (multi-
% plicaci´on de electrones), de tal manera que un solo electr´on puede generar varios
% ´ordenes de magnitud de electrones secundarios (ganancia). Los electrones mul-
% tiplicados son recogidos en un conjunto de electrodos o en una segunda capa de
% multiplicaci´on, resultando una se˜nal de corriente el´ectrica que puede ser medida
% y analizada.

\end{document}