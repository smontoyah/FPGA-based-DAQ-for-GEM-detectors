\documentclass[]{book}
\usepackage[english]{babel}
\begin{document}


\chapter{Introducción}

\subsection{Motivación}

Desde los inicios de la humanidad, el conocimiento sobre el mundo ha surgido de la observación del entorno. Durante milenios, el ser humano ha utilizado sus sentidos para obtener información sobre las características de su alimento, el clima, sus depredadores y muchos otros aspectos fundamentales para su supervivencia. Sin embargo, con el desarrollo de sus habilidades cognitivas, esta capacidad de observación evolucionó hacia preguntas sobre la naturaleza de las cosas, su composición y origen, dando lugar a la actividad científica.\\

% \noindent El entendimiento de la visión, en particular, como el fenómeno por el cual la luz reflejada por los objetos es captada por el ojo, proporcionó una noción fundamental: es necesaria una interacción con el objeto para poder observarlo. Sin embargo, al explorar la estructura básica de la materia, la limitada capacidad de visión no permitía discernir sus componentes, incluso con los avances en tecnología óptica.\\

\noindent A principios del siglo XX, la idea del átomo como unidad estructural de la materia llevó a científicos como Thomson y Rutherford a interactuar con nuevos tipos de radiación descubiertos, como los rayos catódicos y las partículas alfa, revelando la naturaleza eléctrica y la distribución interna del átomo, respectivamente. Estos avances fueron posibles gracias a montajes experimentales basados en sistemas de detección especializados, diseñados sistemáticamente para interactuar con las partículas incidentes y proporcionar información sobre la respuesta del objeto bajo inspección. \\

% Así, la física de las interacciones entre partículas se utiliza como medio para desarrollar métodos y dispositivos de detección, los cuales a su vez tienen aplicaciones de importancia, desde mejorar la comprensión de la naturaleza fundamental de la materia, hasta dispositivos médicos de diagnóstico y variedad de bienes de consumo [Knoll].\\

% \noindent Es importante resaltar que las partículas de interés guían el diseño de un detector. No obstante, el principio básico de funcionamiento de diversos tipos, como los detectores gaseosos, líquidos o de estado sólido, se rige por la interacción electromagnética, ya que todas las partículas, a excepción de los neutrones y los neutrinos, se ven afectadas por ella. Incluso en el caso específico de estas partículas neutras, se aprovechan otras interacciones físicas como las nucleares, que resultan en fenómenos electromagnéticos, permitiendo el uso de los dispositivos mencionados. Esta característica hace que la mayoría los sistemas de detección modernos cuenten con una etapa que transforma la información física de interés, producto de la detección, en señales eléctricas. Adicionalmente, gracias al amplio desarrollo y uso de dispositivos electrónicos, estas señales se modulan, transportan, procesan y almacenan con alta precisión.\\

\noindent A grandes rasgos, un sistema de detección de partículas está compuesto por un detector que utiliza una o varias interacciones de la partícula objetivo para generar un evento de detección en forma de señal eléctrica. En segundo lugar, la electrónica de lectura, que se puede dividir en una etapa analógica denominada electrónica de front-end para el acondicionamiento de la señal y en una etapa digital que compone al sistema de adquisición procesamiento de datos. Finalmente, se dispone de una infraestructura para la transmisión de señales eléctricas, ópticas o híbridas hacia subsecuentes sistemas de procesamiento o almacenamiento [Kolanoski].\\

% \noindent En particular, los detectores GEM (Gas Electron Multiplier), que son el foco de este estudio, consisten en una cámara llena de gas cuyas moléculas pueden ser ionizadas por partículas cargadas que la atraviesan. Dentro de esta cavidad, se encuentra una estructura compuesta por láminas de material dieléctrico perforadas con agujeros micrométricos dispuestos en patrones específicos y recubiertas de material conductor en ambas caras. Al aplicar una diferencia de potencial eléctrico a estas láminas, actúan como etapas multiplicadoras de los electrones generados por la ionización del gas, permitiendo que la carga sea colectada por electrodos conectados a la electrónica de lectura [Estrada].\\

\noindent Sin embargo, la electrónica comercial dedicada a la lectura de detectores suele ser costosa, voluminosa y de capacidades limitadas. Por ello, este trabajo se centra en la implementación y prueba de un sistema de detección para un detector GEM, con un enfoque en la adquisición, procesamiento digital y comunicación de señales, empleando tecnología SoC con hardware configurable en FPGA y un procesador embebido. El objetivo es desarrollar una plataforma accesible y flexible para la experimentación en física nuclear y de partículas.

% sistema de adquisición de datos al montaje experimental que incluye las etapas de acondicionamiento de señales y procesamiento digital, las cuales permiten la lectura de las señales eléctricas resultantes de las avalanchas de electrones generadas en el detector. Utilizando electrónica analógica para la preamplificación y amplificación, un conversor análogo a digital (ADC) de alto rendimiento de un canal, una plataforma de hardware configurable para procesamiento digital tipo SoC (que incluye una FPGA y un procesador), y una interfaz de comunicación de alta velocidad con la computadora, se adquirirán, procesarán y visualizarán las señales provenientes de un detector GEM en varias configuraciones.

\subsection{Objetivos}
\noindent Como meta principal se propone diseñar y desarrollar un sistema de adquisición de datos basado en FPGA para detectores GEM. Para lograr esto, se plantean los siguientes objetivos específicos:

\begin{itemize}
    \item Caracterizar los requerimientos técnicos del sistema de adquisición de datos (DAQ) en función del sistema de detección
    \item Seleccionar el hardware adecuado para cumplir con los requerimientos del DAQ
    \item Programar el hardware, firmware y software para la adquisición y procesamiento de datos de acuerdo a las necesidades del experimento 
    \item Implementar la comunicación entre las distintas etapas del DAQ 
    \item Diseñar y programar una interfaz de usuario acorde a las necesidades del experimento para la interacción con el DAQ
    
\end{itemize}

\subsection{Estructura de la tesis}

\noindent Este trabajo se divide en capítulos. Inicialmente, en el capítulo de teoría se abordan tanto los principales conceptos de la física de los detectores GEM, como una revisión general de la electrónica involucrada en un DAQ en el contexto de la instrumentación para detectores. En segundo lugar, en el capítulo de sistema, se realiza una descripción y caracterización general de los dispositivos utilizados en la cadena de lectura de señales, así como de su integración en un sistema completo de detección de partículas. A continuación, se presentan los resultados experimentales que incluyen la calibración del DAQ y su respectivo análisis. Finalmente, se exponen las conclusiones y perspectivas del trabajo.


\end{document}
