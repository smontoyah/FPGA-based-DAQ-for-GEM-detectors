\documentclass[]{book}
\usepackage[english]{babel}
\begin{document}


\chapter{Introducción}
Desde los inicios de la humanidad, el conocimiento sobre el mundo ha surgido de la observación del entorno. 
Durante milenios, el ser humano ha utilizado sus sentidos para obtener información sobre las características 
de su alimento, el clima, sus depredadores y muchos otros aspectos fundamentales para su supervivencia. 
Con el desarrollo de sus habilidades cognitivas, esta capacidad de observación evolucionó hacia preguntas
sobre la naturaleza de las cosas, su composición y origen, dando lugar a la actividad científica.
El entendimiento de la visión, en particular, como el fenómeno por el cual la luz reflejada por los objetos 
es captada por el ojo, proporcionó una noción fundamental: es necesaria una interacción con el objeto para 
poder observarlo. Sin embargo, al explorar la estructura de la materia, la limitada capacidad de visión no 
permitía discernir los componentes básicos. Incluso con los avances en la tecnología óptica, la interacción
de la luz visible con los objetos resultaba insuficiente.
Gracias a los avances en física, se logró describir otras interacciones fundamentales de la materia, además 
de la radiación electromagnética (de la cual la luz visible es un tipo). Estos descubrimientos abrieron nuevas
posibilidades para la observación y detección, facilitando el desarrollo de técnicas y herramientas que nos 
permiten explorar y comprender mejor el universo a niveles antes inimaginables.

La física de los detectores está estrechamente vinculada con diversas áreas de la física y la ingeniería. 
El desarrollo y funcionamiento de estos dispositivos demandan conocimientos sobre las interacciones de las 
partículas con la materia, así como sobre la física de gases, líquidos y sólidos. Además, es esencial comprender 
los fenómenos de transporte de cargas y la generación de señales, así como dominar las técnicas de procesamiento 
de señales electrónicas y la microelectrónica.

Sin embargo




\end{document}
