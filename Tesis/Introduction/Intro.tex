\documentclass[]{book}
\usepackage[english]{babel}
\begin{document}


\chapter{Introducción}

Desde los inicios de la humanidad, el conocimiento sobre el mundo ha surgido de la observación del entorno. Durante milenios, el ser humano ha utilizado sus sentidos para obtener información sobre las características de su alimento, el clima, sus depredadores y muchos otros aspectos fundamentales para su supervivencia. Con el desarrollo de sus habilidades cognitivas, esta capacidad de observación evolucionó hacia preguntas sobre la naturaleza de las cosas, su composición y origen, dando lugar a la actividad científica.

\noindent El entendimiento de la visión, en particular, como el fenómeno por el cual la luz reflejada por los objetos es captada por el ojo, proporcionó una noción fundamental: es necesaria una interacción con el objeto para poder observarlo. Sin embargo, al explorar la estructura básica de la materia, la limitada capacidad de visión no permitía discernir sus componentes, incluso con los avances en tecnología óptica.

A principios del siglo XX, la idea del átomo como unidad estructural de la materia llevó a científicos como Thomson y Rutherford a interactuar con nuevos tipos de radiación descubiertos, como los rayos catódicos y las partículas alfa, revelando la naturaleza eléctrica y la distribución interna del átomo, respectivamente. Estos avances fueron posibles gracias a montajes experimentales basados en sistemas de detección especializados, diseñados sistemáticamente para interactuar con las partículas incidentes y proporcionar información física sobre la respuesta del sistema bajo inspección.

Es así que, el desarrollo de métodos de detección

Gracias a los avances en física, se logró describir otras interacciones fundamentales de la materia, además 
de la radiación electromagnética (de la cual la luz visible es un tipo) como las fuerzas nucleares y la gravedad, llegando a un modelo teórico que explica la esctructura de la materia en  
Estos descubrimientos abrieron nuevas posibilidades para la observación y detección, facilitando el desarrollo de técnicas y herramientas que nos permiten explorar y comprender mejor el universo a niveles antes inimaginables.

La física de los detectores está estrechamente vinculada con diversas áreas de la física y la ingeniería. 
El desarrollo y funcionamiento de estos dispositivos demandan conocimientos sobre las interacciones de las 
partículas con la materia, así como sobre la física de gases, líquidos y sólidos. Además, es esencial comprender 
los fenómenos de transporte de cargas y la generación de señales, así como dominar las técnicas de procesamiento 
de señales electrónicas y la microelectrónica.

Sin embargo




\end{document}
