\documentclass{article}

\title{\textbf{Capítulo introductorio}}
\author{Sebastián Montoya Hernández}



\usepackage[english]{babel}
\usepackage{graphicx}
\usepackage{amsmath}
\usepackage{cite}
\usepackage{hyperref}
\usepackage{caption}
\usepackage{float}
\usepackage{listings}

\begin{document}

\maketitle 
\setcounter{section}{0}

\noindent Desde tiempos inmemoriales, el ser humano ha utilizado sus sentidos para interpretar el entorno, siendo la visión uno de los más fundamentales. La percepción visual se basa en la interacción de la luz, o de forma más general, la radiación electromagnética, con la materia. La luz se dispersa al incidir sobre los objetos, es absorbida por el ojo y transformada en señales neuronales que el cerebro procesa para generar la imagen percibida. De manera análoga, la detección de partículas elementales, núcleos y radiación electromagnética de alta energía, comúnmente denominadas "partículas", también depende de su interacción con la materia. \\

\noindent El estudio de partículas subatómicas comenzó a principios del siglo XX, cuando científicos como J.J. Thomson y Ernest Rutherford realizaron experimentos pioneros que revelaron la estructura interna del átomo. Thomson, con su descubrimiento del electrón a través de la manipulación de los rayos catódicos, y Rutherford, al desentrañar la estructura nuclear mediante la dispersión de partículas alfa, sentaron las bases de la física atómica moderna. Estos experimentos fueron posibles gracias al uso de sistemas de detección que permitieron observar indirectamente los efectos de las partículas sobre la materia. La creación de estos montajes experimentales, que incluían detectores rudimentarios pero eficaces, supuso un avance crucial para la ciencia, ya que facilitó la interacción controlada con partículas incidentes, proporcionando información valiosa sobre la estructura de la materia. \\

\noindent A medida que avanzaba el siglo XX, la necesidad de detectar partículas subatómicas de manera más precisa y en entornos más complejos se hizo cada vez más urgente. A diferencia de la luz visible, las partículas no son perceptibles directamente por nuestros órganos sensoriales, lo que llevó al desarrollo de dispositivos especializados llamados detectores, en los cuales las partículas interactúan con la materia, generando señales que los científicos pueden analizar. Estos detectores explotan principalmente la interacción electromagnética para registrar la presencia de partículas. Por ejemplo, las partículas cargadas producen ionización a lo largo de su trayectoria, mientras que los fotones y electrones generan cascadas electromagnéticas que permiten medir su energía. Incluso otras interacciones, como la fuerte y la débil, son aprovechadas para detectar neutrones y neutrinos, respectivamente. \\

\noindent El desarrollo de detectores y métodos de detección ha sido impulsado en gran medida por la necesidad de aplicaciones en la física de partículas y nuclear. Estos campos dependen de la capacidad para medir no solo la presencia de partículas, sino también sus propiedades cinemáticas como la dirección, el momento y la energía, así como determinar su identidad. Los primeros dispositivos, como las cámaras de niebla de C.T.R. Wilson o las cámaras de burbujas de Donald Glaser, revolucionaron la forma en que los científicos podían observar eventos a nivel subatómico. A lo largo del tiempo, estas tecnologías se fueron perfeccionando, lo que permitió detectar partículas con mayor precisión y en un rango más amplio de energías.\\

\noindent Con el tiempo, surgieron diferentes tipos de detectores, cada uno aprovechando distintos fenómenos físicos. Los detectores de estado sólido, por ejemplo, son muy utilizados por su alta resolución espacial y energética, mientras que los detectores de centelleo, basados en cristales como el NaI o CsI, convierten la energía de la radiación en luz que luego puede ser medida. Además, los detectores líquidos, como los de argón líquido, ofrecen una excelente resolución en la detección de neutrinos y partículas cargadas en grandes volúmenes. Existen también sistemas híbridos que combinan múltiples tecnologías para explotar diversos mecanismos de interacción de partículas, como la ionización y la emisión de radiación secundaria. Dentro de esta amplia gama, los detectores gaseosos han desempeñado un papel clave debido a su capacidad para cubrir grandes áreas con una cantidad mínima de material y aprovechar la ionización de partículas en el gas para generar señales medibles. \\

\noindent Un ejemplo destacado de esta evolución en los detectores gaseosos es la tecnología de los Gas Electron Multipliers (GEMs), basada en láminas perforadas que amplifican las señales resultantes de la ionización del gas. Esta tecnología ofrece una mayor sensibilidad y capacidad para manejar altas tasas de conteo, haciéndola ideal para experimentos de alta precisión. Además, los GEMs son especialmente eficaces en entornos con campos magnéticos, ya que permiten rastrear con precisión las trayectorias de partículas cargadas. En comparación con los detectores de semiconductores, los GEMs son más económicos y ligeros, lo que los convierte en una opción preferida cuando es necesario cubrir grandes volúmenes.\\

\noindent En este capítulo se examinan los principios fundamentales de la física que gobiernan el funcionamiento e interacción de los detectores gaseosos, revisando conceptos clave como la ionización, recombinación, poder de frenado y amplificación. Además, se estudian diferentes tipos de detectores gaseosos, desde los contadores de eventos basados en placas paralelas hasta las cámaras de múltiples hilos (MWPCs), culminando en los GEMs. También se explican las señales típicas que estos detectores generan y la instrumentación necesaria para su acondicionamiento y adquisición, que constituyen el objeto central de este trabajo.


\subsection*{References}
\begin{enumerate}
    \item Sauli, F. (2015). Gaseous radiation detectors: fundamentals and applications (p. 497). Cambridge University Press.
    \item Giovani Mocellin. “Performance of the GE1/1 detectors for the upgrade of the
    CMS Muon Forward system”. PhD thesis. Rheinish-Westf¨alische Technische
    Hochschule Aachen University, 2021. url: https://cds.cern.ch/record/
    2809098.
    \item K Nakamura and (Particle Data Group) 2010 J. Phys. G: Nucl. Part. Phys. 37 075021. DOI 10.1088/0954-3899/37/7A/075021
\end{enumerate}



% Así, la física de las interacciones entre partículas se utiliza como medio para desarrollar métodos y dispositivos de detección, los cuales a su vez tienen aplicaciones de importancia, desde mejorar la comprensión de la naturaleza fundamental de la materia, hasta dispositivos médicos de diagnóstico y variedad de bienes de consumo [Knoll].\\

% \noindent Es importante resaltar que las partículas de interés guían el diseño de un detector. No obstante, el principio básico de funcionamiento de diversos tipos, como los detectores gaseosos, líquidos o de estado sólido, se rige por la interacción electromagnética, ya que todas las partículas, a excepción de los neutrones y los neutrinos, se ven afectadas por ella. Incluso en el caso específico de estas partículas neutras, se aprovechan otras interacciones físicas como las nucleares, que resultan en fenómenos electromagnéticos, permitiendo el uso de los dispositivos mencionados. Esta característica hace que la mayoría los sistemas de detección modernos cuenten con una etapa que transforma la información física de interés, producto de la detección, en señales eléctricas. Adicionalmente, gracias al amplio desarrollo y uso de dispositivos electrónicos, estas señales se modulan, transportan, procesan y almacenan con alta precisión.\\

% \noindent A grandes rasgos, un sistema de detección de partículas está compuesto por un detector que utiliza una o varias interacciones de la partícula objetivo para generar un evento de detección en forma de señal eléctrica. En segundo lugar, la electrónica de lectura, que se puede dividir en una etapa analógica denominada electrónica de front-end para el acondicionamiento de la señal y en una etapa digital que compone al sistema de adquisición (DAQ) y procesamiento de datos. Finalmente, se dispone de una infraestructura para la transmisión de señales eléctricas, ópticas o híbridas hacia subsecuentes sistemas de procesamiento o almacenamiento [Kolanoski].\\

% \noindent En particular, los detectores GEM (Gas Electron Multiplier), que son el foco de este estudio, consisten en una cámara llena de gas cuyas moléculas pueden ser ionizadas por partículas cargadas que la atraviesan. Dentro de esta cavidad, se encuentra una estructura compuesta por láminas de material dieléctrico perforadas con agujeros micrométricos dispuestos en patrones específicos y recubiertas de material conductor en ambas caras. Al aplicar una diferencia de potencial eléctrico a estas láminas, actúan como etapas multiplicadoras de los electrones generados por la ionización del gas, permitiendo que la carga sea colectada por electrodos conectados a la electrónica de lectura [Estrada].\\


% sistema de adquisición de datos al montaje experimental que incluye las etapas de acondicionamiento de señales y procesamiento digital, las cuales permiten la lectura de las señales eléctricas resultantes de las avalanchas de electrones generadas en el detector. Utilizando electrónica analógica para la preamplificación y amplificación, un conversor análogo a digital (ADC) de alto rendimiento de un canal, una plataforma de hardware configurable para procesamiento digital tipo SoC (que incluye una FPGA y un procesador), y una interfaz de comunicación de alta velocidad con la computadora, se adquirirán, procesarán y visualizarán las señales provenientes de un detector GEM en varias configuraciones.



% \subsection{Estructura de la tesis}

% \noindent Este trabajo se divide en capítulos. Inicialmente, en el capítulo de teoría, se abordan tanto los principales conceptos de la física de los detectores GEM, como una revisión general de la electrónica involucrada en un DAQ en el contexto de la instrumentación para detectores. En segundo lugar, en el capítulo de sistema, se realiza una descripción y caracterización de los dispositivos utilizados en la cadena de lectura de señales, así como de su integración en el sistema completo de detección. A continuación, se presentan los resultados experimentales que incluyen la calibración del DAQ y su respectivo análisis. Finalmente, se exponen las conclusiones y perspectivas del trabajo.

% \subsection{Objetivos}
% \noindent Como meta principal se propone diseñar y desarrollar un sistema de adquisición de datos basado en FPGA para detectores GEM. Para lograr esto, se plantean los siguientes objetivos específicos:

% \begin{itemize}
%     \item Caracterizar los requerimientos técnicos del sistema de adquisición de datos (DAQ) en función del sistema de detección
%     \item Seleccionar el hardware adecuado para cumplir con los requerimientos del DAQ
%     \item Programar el hardware, firmware y software para la adquisición y procesamiento de datos de acuerdo a las necesidades del experimento 
%     \item Implementar la comunicación entre las distintas etapas del DAQ 
%     \item Diseñar y programar una interfaz de usuario acorde a las necesidades del experimento para la interacción con el DAQ
    


\end{document}
