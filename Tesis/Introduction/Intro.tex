\documentclass[]{book}
\usepackage[english]{babel}
\begin{document}


\chapter{Introducción}

Desde los inicios de la humanidad, el conocimiento sobre el mundo ha surgido de la observación del entorno. Durante milenios, el ser humano ha utilizado sus sentidos para obtener información sobre las características de su alimento, el clima, sus depredadores y muchos otros aspectos fundamentales para su supervivencia. Sin embargo, con el desarrollo de sus habilidades cognitivas, esta capacidad de observación evolucionó hacia preguntas sobre la naturaleza de las cosas, su composición y origen, dando lugar a la actividad científica.

\noindent El entendimiento de la visión, en particular, como el fenómeno por el cual la luz reflejada por los objetos es captada por el ojo, proporcionó una noción fundamental: es necesaria una interacción con el objeto para poder observarlo. Sin embargo, al explorar la estructura básica de la materia, la limitada capacidad de visión no permitía discernir sus componentes, incluso con los avances en tecnología óptica.

\noindent A principios del siglo XX, la idea del átomo como unidad estructural de la materia llevó a científicos como Thomson y Rutherford a interactuar con nuevos tipos de radiación descubiertos, como los rayos catódicos y las partículas alfa, revelando la naturaleza eléctrica y la distribución interna del átomo, respectivamente. Estos avances fueron posibles gracias a montajes experimentales basados en sistemas de detección especializados, diseñados sistemáticamente para interactuar con las partículas incidentes y proporcionar información sobre la respuesta del objeto bajo inspección. Así, la física de las interacciones entre partículas se utiliza como medio para desarrollar métodos y dispositivos de detección, los cuales a su vez tienen aplicaciones de importancia, desde mejorar la comprensión de la naturaleza fundamental de la materia, hasta dispositivos médicos de diagnóstico y variedad de bienes de consumo.

\noindent Es importante resaltar que las partículas de interés guían el diseño de un detector. No obstante, el principio básico de funcionamiento de diversos tipos, como los detectores gaseosos, líquidos o de estado sólido, se rige por la interacción electromagnética, ya que todas las partículas, a excepción de los neutrones y los neutrinos, se ven afectadas por ella. Incluso en el caso específico de estas partículas neutras, se aprovechan otras interacciones físicas como las nucleares, que resultan en fenómenos electromagnéticos, permitiendo el uso de los dispositivos mencionados. Esta característica hace que todos los sistemas de detección modernos cuenten con una etapa que transforma la información física de interés, producto de la detección, en señales eléctricas. Adicionalmente, gracias al amplio desarrollo y uso de dispositivos electrónicos, estas señales se modulan, transportan, procesan y almacenan con alta precisión.

\noindent Un sistema de detección de partículas está compuesto por un detector que utiliza una o varias interacciones de la partícula objetivo para generar un evento de detección en forma de señal eléctrica. Luego, una etapa de electrónica de acondicionamiento de señales, que puede incluir subetapas de preamplificación, se diseña según las necesidades específicas del sistema. Además, cuenta con un sistema de adquisición de datos que puede ser digital, analógico o híbrido, seguido por una etapa de procesamiento digital. Finalmente, se dispone de una infraestructura para la transmisión de señales eléctricas, ópticas o híbridas [Kolanoski].

\noindent En particular, los detectores GEM, que son el foco de este trabajo, consisten en una cámara llena de gas cuyas moléculas pueden ser ionizadas por partículas cargadas que pasan a través de ella. Dentro de esta cavidad, hay una estructura formada por láminas de material dieléctrico perforadas con agujeros micrométricos dispuestos en patrones específicos y recubiertas con material conductor en ambas caras. Cuando se aplica una diferencia de potencial eléctrico a estas láminas, actúan como etapas multiplicadoras de electrones generados por el efecto de ionización del gas [Estrada].


\end{document}
