\documentclass[]{book}
\usepackage[english]{babel}
\begin{document}


\chapter{Introducción}

Desde los inicios de la humanidad, el conocimiento sobre el mundo ha surgido de la observación del entorno. Durante milenios, el ser humano ha utilizado sus sentidos para obtener información sobre las características de su alimento, el clima, sus depredadores y muchos otros aspectos fundamentales para su supervivencia. Sin embargo, con el desarrollo de sus habilidades cognitivas, esta capacidad de observación evolucionó hacia preguntas sobre la naturaleza de las cosas, su composición y origen, dando lugar a la actividad científica.

\noindent El entendimiento de la visión, en particular, como el fenómeno por el cual la luz reflejada por los objetos es captada por el ojo, proporcionó una noción fundamental: es necesaria una interacción con el objeto para poder observarlo. Sin embargo, al explorar la estructura básica de la materia, la limitada capacidad de visión no permitía discernir sus componentes, incluso con los avances en tecnología óptica.

\noindent A principios del siglo XX, la idea del átomo como unidad estructural de la materia llevó a científicos como Thomson y Rutherford a interactuar con nuevos tipos de radiación descubiertos, como los rayos catódicos y las partículas alfa, revelando la naturaleza eléctrica y la distribución interna del átomo, respectivamente. Estos avances fueron posibles gracias a montajes experimentales basados en sistemas de detección especializados, diseñados sistemáticamente para interactuar con las partículas incidentes y proporcionar información sobre la respuesta del objeto bajo inspección. Así, la física de las interacciones entre partículas se utiliza como medio para desarrollar métodos y dispositivos de detección, los cuales a su vez tienen aplicaciones de importancia, desde mejorar la comprensión de la naturaleza fundamental de la materia, hasta dispositivos médicos de diagnóstico y variedad de bienes de consumo.

\noindent Es importante resaltar que las partículas de interés guían el diseño de un detector. No obstante, el principio básico de funcionamiento de diversos tipos de detectores se rige por la interacción electromagnética, ya que todas las partículas, a excepción de los neutrones y neutrinos, se ven afectadas por ella. Esta característica hace que todos los sistemas de detección modernos, cuenten con una etapa que transforma la información de interés producto de la detección en señales eléctricas, que, gracias al amplio desarrollo y uso de dispositivos electrónicos, la modulan, transportan, procesan y almacenan con alta precisión. 

\noindent 

\noindent La física de los detectores está estrechamente vinculada con diversas áreas de la física y la ingeniería. 
El desarrollo y funcionamiento de estos dispositivos demandan conocimientos sobre las interacciones de las 
partículas con la materia, así como sobre la física de gases, líquidos y sólidos. Además, es esencial comprender 
los fenómenos de transporte de cargas y la generación de señales, así como dominar las técnicas de procesamiento 
de señales electrónicas y la microelectrónica.

\noindent 


\end{document}
