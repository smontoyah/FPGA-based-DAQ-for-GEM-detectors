\documentclass{article}

\title{\textbf{Capítulo introductorio}}
\author{Sebastián Montoya Hernández}



\usepackage[english]{babel}
\usepackage{graphicx}
\usepackage{amsmath}
\usepackage{cite}
\usepackage{hyperref}
\usepackage{caption}
\usepackage{float}
\usepackage{listings}

\begin{document}

\maketitle 
\setcounter{section}{0}

\noindent Desde tiempos inmemoriales, los seres humanos han utilizado sus sentidos para interpretar y comprender el entorno. La visión, uno de los sentidos más fundamentales, permite la interacción con el mundo a través de la captación de la luz, o más ampliamente, de la radiación electromagnética. Esta luz interactúa con la materia, se dispersa o es absorbida, y es procesada por nuestros ojos para generar señales neuronales que el cerebro traduce en imágenes. Sin embargo, la detección no se limita solo a los sentidos humanos. A lo largo de la historia, el desarrollo de dispositivos y tecnologías ha expandido nuestra capacidad para percibir fenómenos imperceptibles para nuestros sentidos, como las partículas subatómicas y la radiación electromagnética de alta energía \cite{swets1960signal}. De manera análoga a nuestros ojos, estos dispositivos, los detectores, permiten la interacción controlada entre las partículas y la materia, proporcionando información valiosa para entender los secretos del universo.\\

\noindent El estudio de partículas subatómicas cobró relevancia a inicios del siglo XX, con experimentos pioneros realizados por científicos como J.J. Thomson \cite{smith2001jj} y Ernest Rutherford\cite{rutherford1920bakerian}. Thomson descubrió el electrón mediante el uso de rayos catódicos, y Rutherford desentrañó la estructura del núcleo atómico al estudiar la dispersión de partículas alfa. Estos experimentos, que marcaron el nacimiento de la física atómica moderna, fueron posibles gracias al uso de detectores rudimentarios pero innovadores, que permitieron observar los efectos de partículas invisibles sobre la materia. Estos avances científicos evidenciaron la necesidad de desarrollar tecnologías más sofisticadas para detectar partículas en entornos cada vez más complejos.\\

\noindent A medida que la física de partículas avanzaba, la detección de partículas subatómicas requirió el desarrollo de dispositivos cada vez más especializados, capaces de explotar diversos fenómenos físicos como la ionización de gases, la emisión de luz en cristales (centelleo) o la generación de cascadas electromagnéticas \cite{kolanoski2020particle}. Uno de los primeros avances clave en esta área fue la creación de la cámara de ionización, que permitía detectar partículas midiendo la ionización producida en un gas. Este diseño evolucionó hacia el contador proporcional, fruto del trabajo pionero de Rutherford y su colaborador Hans Geiger \cite{friedman1949geiger}. Los contadores proporcionales no solo mejoraban la detección al amplificar las señales generadas por la ionización, sino que también sentaron las bases para dispositivos icónicos como el contador Geiger-Müller, utilizado durante décadas para la detección de radiación ionizante. No obstante, con el aumento de los requerimientos experimentales, surgieron nuevas tecnologías como las cámaras de ionización de múltiples hilos (Multi-Wire Proportional Chambers, MWPC), que ofrecieron una mayor resolución espacial y capacidad de registro en experimentos de mayor escala y complejidad \cite{charpak1979multiwire}.\\

\noindent Una de las evoluciones más recientes y sofisticadas en esta línea de detectores gaseosos es la tecnología de los detectores de micropatrones, entre los cuales destaca el Gas Electron Multiplier (GEM) \cite{sauli1997gem}. Los GEMs, basados en láminas perforadas que amplifican las señales de ionización, permiten una mayor sensibilidad y una excelente capacidad para manejar altas tasas de conteo. Además, son especialmente útiles en entornos con campos magnéticos intensos, ya que pueden rastrear con gran precisión las trayectorias de partículas cargadas. Los GEMs presentan ventajas significativas frente a los detectores de semiconductores: son más ligeros, económicos y pueden cubrir grandes áreas, lo que los convierte en la tecnología ideal tanto para experimentos como aplicaciones de gran envergadura y alto rendimiento \cite{sauli1999micropattern}.\\

\noindent La utilización de detectores GEM requiere sistemas de adquisición de datos robustos y eficientes, ya que no solo deben manejar grandes volúmenes de datos, sino también hacerlo con alta precisión y en tiempo real, garantizando que se preserven las características espaciales y temporales de las señales emitidas por los detectores \cite{cheng2012multi}. La estructura típica de un sistema de adquisición de datos se basa en una serie de etapas electrónicas de acondicionamiento de la señal y plataformas de procesamiento digital, las cuales son esenciales para extraer información útil de las señales eléctricas generadas en los detectores y transferirlas al usuario para su análisis \cite{kolanoski2020particle2}.\\

\noindent Debido a que las señales producidas por los GEMs son de baja amplitud respecto a los parámetros de funcionamiento de la electrónica convencional, es necesario el uso de preamplificadores de carga, dispositivos encargados de amplificar estas pequeñas señales sin introducir ruido adicional, conservando la integridad de la señal y permitiendo su posterior procesamiento \cite{pezzotta2015design}. Una vez amplificadas, las señales deben ser optimizadas a través de shapers, cuya función es mejorar la relación señal-ruido y adecuar la forma de la señal para su correcta digitalización. Estos procesos, que pueden realizarse por medio de circuitos discretos o en ASICs especializados \cite{aspell2005vfat2}, permiten destacar características clave de las señales como el tiempo de llegada y la amplitud, asegurando que la información física de interés se conserve para su análisis posterior.\\

\noindent El desafío de procesar grandes volúmenes de datos en paralelo y a alta velocidad recae en las plataformas de procesamiento digital, como las FPGAs (Field Programmable Gate Arrays) \cite{choubey2006high}. Las FPGAs permiten la implementación de algoritmos complejos de procesamiento de señales en tiempo real, lo que resulta esencial para los sistemas de adquisición de datos de detectores GEM \cite{yu2005gem}. Además, gracias a su capacidad de ser reconfiguradas dinámicamente, las FPGAs pueden adaptarse a diferentes experimentos o condiciones, optimizando el rendimiento del sistema sin necesidad de hardware adicional.\\

\noindent En este trabajo se realiza una revisión general desde la instrumentación científica sobre las características, estructura y ejemplos de un sistema de adquisición de datos, fundamental para la medición de magnitudes físicas. Se pone especial énfasis en los sistemas de detección de radiación y partículas, describiendo algunas particularidades de las señales que producen y la tecnología necesaria para procesarlas. Además, se expone la implementación de un sistema de adquisición de datos monocanal de alta velocidad basado en FPGA SoC, abarcando las etapas de hardware, firmware y software, que incluyen subsistemas de digitalización, procesamiento de señales, interfaces de comunicación de alta velocidad e interfaz de usuario. Se valida el sistema utilizando señales sintéticas que simulan las generadas por un detector de partículas, analizando el comportamiento de cada etapa según los estímulos ingresados. El objetivo es emplear esta plataforma para la lectura de señales de detectores GEM, destacando la flexibilidad y capacidad de reconfiguración que ofrece el desarrollo basado en FPGA, lo cual abre nuevas posibilidades tanto para la investigación básica en física de partículas como para aplicaciones en instrumentación científica.

% \noindent En este capítulo se examinan los principios fundamentales de la física que gobiernan el funcionamiento e interacción de los detectores gaseosos, revisando conceptos clave como la ionización, recombinación, poder de frenado y amplificación. Además, se estudian diferentes tipos de detectores gaseosos, desde los contadores de eventos basados en placas paralelas hasta las cámaras de múltiples hilos (MWPCs), culminando en los GEMs. También se explican las señales típicas que estos detectores generan y la instrumentación necesaria para su acondicionamiento y adquisición, que constituyen el objeto central de este trabajo.

% Así, la física de las interacciones entre partículas se utiliza como medio para desarrollar métodos y dispositivos de detección, los cuales a su vez tienen aplicaciones de importancia, desde mejorar la comprensión de la naturaleza fundamental de la materia, hasta dispositivos médicos de diagnóstico y variedad de bienes de consumo [Knoll].\\

% \noindent Es importante resaltar que las partículas de interés guían el diseño de un detector. No obstante, el principio básico de funcionamiento de diversos tipos, como los detectores gaseosos, líquidos o de estado sólido, se rige por la interacción electromagnética, ya que todas las partículas, a excepción de los neutrones y los neutrinos, se ven afectadas por ella. Incluso en el caso específico de estas partículas neutras, se aprovechan otras interacciones físicas como las nucleares, que resultan en fenómenos electromagnéticos, permitiendo el uso de los dispositivos mencionados. Esta característica hace que la mayoría los sistemas de detección modernos cuenten con una etapa que transforma la información física de interés, producto de la detección, en señales eléctricas. Adicionalmente, gracias al amplio desarrollo y uso de dispositivos electrónicos, estas señales se modulan, transportan, procesan y almacenan con alta precisión.\\

% \noindent A grandes rasgos, un sistema de detección de partículas está compuesto por un detector que utiliza una o varias interacciones de la partícula objetivo para generar un evento de detección en forma de señal eléctrica. En segundo lugar, la electrónica de lectura, que se puede dividir en una etapa analógica denominada electrónica de front-end para el acondicionamiento de la señal y en una etapa digital que compone al sistema de adquisición (DAQ) y procesamiento de datos. Finalmente, se dispone de una infraestructura para la transmisión de señales eléctricas, ópticas o híbridas hacia subsecuentes sistemas de procesamiento o almacenamiento [Kolanoski].\\

% \noindent En particular, los detectores GEM (Gas Electron Multiplier), que son el foco de este estudio, consisten en una cámara llena de gas cuyas moléculas pueden ser ionizadas por partículas cargadas que la atraviesan. Dentro de esta cavidad, se encuentra una estructura compuesta por láminas de material dieléctrico perforadas con agujeros micrométricos dispuestos en patrones específicos y recubiertas de material conductor en ambas caras. Al aplicar una diferencia de potencial eléctrico a estas láminas, actúan como etapas multiplicadoras de los electrones generados por la ionización del gas, permitiendo que la carga sea colectada por electrodos conectados a la electrónica de lectura [Estrada].\\

% sistema de adquisición de datos al montaje experimental que incluye las etapas de acondicionamiento de señales y procesamiento digital, las cuales permiten la lectura de las señales eléctricas resultantes de las avalanchas de electrones generadas en el detector. Utilizando electrónica analógica para la preamplificación y amplificación, un conversor análogo a digital (ADC) de alto rendimiento de un canal, una plataforma de hardware configurable para procesamiento digital tipo SoC (que incluye una FPGA y un procesador), y una interfaz de comunicación de alta velocidad con la computadora, se adquirirán, procesarán y visualizarán las señales provenientes de un detector GEM en varias configuraciones.

% \subsection{Estructura de la tesis}

% \noindent Este trabajo se divide en capítulos. Inicialmente, en el capítulo de teoría, se abordan tanto los principales conceptos de la física de los detectores GEM, como una revisión general de la electrónica involucrada en un DAQ en el contexto de la instrumentación para detectores. En segundo lugar, en el capítulo de sistema, se realiza una descripción y caracterización de los dispositivos utilizados en la cadena de lectura de señales, así como de su integración en el sistema completo de detección. A continuación, se presentan los resultados experimentales que incluyen la calibración del DAQ y su respectivo análisis. Finalmente, se exponen las conclusiones y perspectivas del trabajo.

% \subsection{Objetivos}
% \noindent Como meta principal se propone diseñar y desarrollar un sistema de adquisición de datos basado en FPGA para detectores GEM. Para lograr esto, se plantean los siguientes objetivos específicos:

% \begin{itemize}
%     \item Caracterizar los requerimientos técnicos del sistema de adquisición de datos (DAQ) en función del sistema de detección
%     \item Seleccionar el hardware adecuado para cumplir con los requerimientos del DAQ
%     \item Programar el hardware, firmware y software para la adquisición y procesamiento de datos de acuerdo a las necesidades del experimento 
%     \item Implementar la comunicación entre las distintas etapas del DAQ 
%     \item Diseñar y programar una interfaz de usuario acorde a las necesidades del experimento para la interacción con el DAQ
    

\newpage

\bibliographystyle{ieeetr}
\bibliography{references}

\end{document}
